
\section{Numerical Methods}

%\begin{frame}{It\^{o}'s Formulas}
\begin{frame}{Ito's Formulas}
  	\begin{itemize}
   		\item First version: $$f(B(b))-f(B(a))=\int_{a}^{b}{\frac{\partial f}{\partial B} 				dB}+\int_{a}^{b}{\frac{1}{2} \frac{\partial^2 f}{\partial B^2} dt} $$\\
    		\item Second version: $$f(b,B(b))-f(a,B(a))=\int_{a}^{b}{\frac{\partial f}{\partial B} 			dB}+\int_{a}^{b}{(\frac{\partial f}{\partial s}+\frac{1}{2}\frac{\partial ^2 f}{\partial 		B^2}) ds}$$\\
  		\item  Third version: The Stochastic Chain Rule $$d\theta=\frac{\partial\theta}{\partial 		t}dt+\frac{\partial\theta}{\partial x}f dB+\frac{\partial\theta}{\partial x}g 					dt+\frac{1}{2}\frac{\partial^2\theta}{\partial x^2}f^2dt$$
  	\end{itemize}
\end{frame}

\begin{frame}{Numerical methods}
Integral form of a SDE: 
$$X(t)=X_0+\int_{0}^{t}f(X(s))ds+\int_{0}^{t}g(X(s))dW(s)$$

X(t) is a random variable for each value of t. To apply numerical methods:
	\begin{itemize}
		\item Discretized time steps	
		\item X(t) will be the limit as our stepsize goes to cero.
	\end{itemize}
\end{frame}

\begin{frame}{Euler-Maruyama method}
Let $X_j=X(\tau_j)$ and $\tau_j=j\Delta t$. In the in the interval [0,L] we have:\\
$$X_j=X_{j-1}+f(X_{j-1})\Delta t+g(X_{j-1})(W(\tau_j)-W(\tau_{j-1}))$$
An approximation for:
$$X(\tau_j)=X(\tau_{j-1})+\int_{\tau_{j-1}}^{\tau_{j}}f(X(s))ds+\int_
{\tau_{j-1}}^{\tau_{j}}X(s))dW(s)$$
We have Euler's method in the deterministic case ($g\equiv 0$)
\end{frame}

\begin{frame}
We say that a method has strong order of convergence equal to $\gamma$ if there exists a constant C such that:
$$E|X_n-X(\tau)|\leq C \Delta t^\gamma$$
for any fixed $\tau=n \Delta t \in [0,T]$ and $\Delta t$ sufficiently small.
\bigskip
The Euler-Maruyama method has strong order of convergence of $\gamma=\frac{1}{2}$.
\end{frame}

\begin{frame}{Milstein's method}
To raise the strong order of convergence we apply the stochastic chain rule and truncate the expansion at an appropriate point. 	
	\begin{equation*}
	\begin{split}
	X_{j} & =X_{j-1}+\Delta t f(X_{j-1})+g(X_{j-1})(W(\tau_j)-W(\tau_{j-1}))\\
	  &\quad +\frac{1}{2}g(X_{j-1})g'(X_{j-1})((W(\tau_j)-W(\tau_{j-1}))^2-\Delta t)
	\end{split}
	\end{equation*}
\end{frame}

\begin{frame}
	\begin{center}
	\includegraphics[scale=0.55]{alfam15_beta05.png} 
	\end{center}
\end{frame}

\begin{frame}
For systems of equations:
	\begin{eqnarray*}
		dx&=&f_1(x,y)dt+g_1(x,y)dW\\
		dy&=&f_2(x,y)dt+g_2(x,y)dW\\
	\end{eqnarray*}

	\begin{eqnarray*}
		x_t=x_a+\int_{a}^{t}f_1 ds+\int_{a}^{t}g_1dW_1\\
		y_t=y_a+\int_{a}^{t}f_2 ds+\int_{a}^{t}g_2dW_2\\
	\end{eqnarray*}	
\end{frame}

\begin{frame}
If we apply the chain rule for $f_1$:
	\begin{equation*}
	\begin{split}
	f_1(t) &=f_1(a)+f_{1x}(a)(x-x_a)+f_{1y}(a)(y-y_a)+\frac{1}{2} f_{1xx}(a)(x-x_a)^2\\
		   &\quad +f_{1xy}(a)(x-x_a)(y-y_a)+\frac{1}{2}f_{1yy}(a)(y-y_a)^2
	\end{split}
	\end{equation*}
From the integral form:
	\begin{equation*}
	x-x_a=\int_{a}^{t}f_1 ds+\int_{a}^{t}g_1dW_1
	\end{equation*}

\end{frame}

\begin{frame}
Similarly for $g_1$ and ignoring the high order terms,we end up with:
\begin{equation*}
x(t)=x(a)+f_1(a)(t-a)+g_1(a)(W_1(t)-W_1(a))+g_{1x}(a)\int_{a}^{t} \int_{a}^{s}g_1dW_1dW_1+g_{iy}(a)\int_{a}^{t}\int_{a}^{t} g_2dW_2dW_1
\end{equation*}

\end{frame}


