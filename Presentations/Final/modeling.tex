\section{Modeling Coral Reef Dynamics}

\begin{frame}
\frametitle{Coral Reef Dynamics}
\includegraphics[scale=.175]{./coral-reef-triangle.png}
\end{frame}

\begin{frame}\frametitle{Coral Reef Dynamics}
The deterministic ordinary differential equation model:
$$\begin{cases}\begin{array}{rl}
\frac{dM}{dt}\hspace{-.8em}&=aMC - \frac{gM}{M+T} + \gamma M T, \\
\frac{dC}{dt}\hspace{-.8em}&=rTC - dC - aMC, \\
\frac{dT}{dt}\hspace{-.8em}&=\frac{gM}{M + T} - \gamma MT - rTC + dC. 
\end{array}\end{cases}$$ 

where 
\begin{itemize}\itemsep0pt
\item $r$ is the rate corals overgrow upon algal turfs\\
\item $d$ is the mortality rate of corals\\
\item $a$ is the rate that macroalgae overgrow upon corals\\
\item $\gamma$ is the rate that macroalgae spread over algal turfs\\
\item $g$ is the indiscriminate grazing rate of parrotfish.
\end{itemize}
\end{frame}

\begin{frame}\frametitle{Coral Reef Dynamics}

\hspace{1.57em}

\begin{itemize}
\item $\frac{dT}{dt}=-\frac{dM}{dt}-\frac{dC}{dt}$ implies $M+C+T$ is
  constant.
\item We assume that $M+C+T=1$.
\item This limits our scope to regions entirely covered by coral,
  macroalgae, and algal turf. 
\end{itemize}

Thus, we reduce our system to, 
$$\begin{cases}
\begin{array}{rl}
\frac{dM}{dt}&= aMC-\frac{gM}{1-C} + \gamma M - \gamma M^2 -\gamma M C,\\
\frac{dC}{dt}&=rC - rC^2 - rCM - dC - aMC.
\end{array} 
\end{cases}$$
\end{frame}


\begin{comment}\frametitle{Equilibria and Stability}
Equilibrium point

\end{comment}

\begin{frame}\frametitle{Equilibria and Stability}
To find equilibrium points analytically we first set the derivatives equal to zero to obtain the nullclines:
$$\begin{cases}
\begin{array}{rl}
0\hspace{-.8em}&=M(aC + \gamma-\gamma M-\gamma C) - \frac{gM}{1-C},\\
0\hspace{-.8em}&=C(r-Mr-Cr - d - aM).\\
\end{array}
\end{cases}$$
\end{frame}



\begin{frame}\frametitle{Equilibria and Stability}

  \begin{itemize}
  \item $M'=0$ is satisfied when:
    \begin{itemize}
    \item $M=0$, or 
    \item
      $aC + \gamma - \gamma M - \gamma C -
      \frac{gM}{1-C}=0$.
    \end{itemize}
  \item $C'=0$ is satisfied when:
    \begin{itemize} 
    \item$C=0$, or 
    \item $r-Mr-Cr-d-aM=0$. 
    \end{itemize}
  \end{itemize}

\end{frame}


\begin{frame}\frametitle{Phase Plane} 

  Our equilibrium points are located at the intersections of the nullclines:\\
  \includegraphics[scale=.2]{./nullclines.png}

\end{frame}



\begin{frame}
\frametitle{Time Delay in the Coral Ecosystem}

We can identify time delays in the coral ecosystem:
\begin{itemize}
\item the delay between overfishing of herbivores and growth of
  macroalgae,
\item the delay between the growth of algae and the effect of algae on
  coral (Jompa, 2003),
\item the delay between the grazing of macroalgae and growth of algal
  turf (Li et al., 2014).
\end{itemize} We focus on the third time delay.

\end{frame}


\begin{frame}\frametitle{Coral Reef Dynamics}
Scarid grazing has an impact for the macroalgae in the future:
$$\begin{cases}
  \begin{array}{rl}
    \frac{dM}{dt}\hspace{-.8em}&=aMC - \frac{gM(t-\tau)}{1-C(t-\tau)} + \gamma M (1-M-C),\\
    \frac{dC}{dt}\hspace{-.8em}&=rC(1-M-C) - dC - aMC,\\
  \end{array}
\end{cases}$$ where $\tau$ is our time delay.
\end{frame}


\begin{frame}
\frametitle{Equilibria and Stability}
There are three equilibrium points of interest: 
\begin{itemize}
\item $(0,0)$ (Unstable for all $\tau\geq0$)\\
\item $(1-\frac{g}{\gamma},0)$ (High Coral Cover)\\
\item $(0,1-\frac{d}{r})$ (High Algae Cover).
\end{itemize} 
Notice that time delay does not affect the equilibria.
\end{frame}


\begin{frame}\frametitle{Jacobian Matrices}
We linearize our delay model:

\begin{eqnarray}
  \label{eqn:linearizedDelayModel}
  \begin{bmatrix} 
    M'\\C'
  \end{bmatrix}=J_1
  \begin{bmatrix} 
    M \\
    C
  \end{bmatrix} + 
  J_2
  \begin{bmatrix}
    M(t-\tau) \\
    C(t-\tau)
  \end{bmatrix},
\end{eqnarray}

where 

\begin{eqnarray*}
  J_1 & = & \begin{bmatrix}
    \gamma-2\gamma M +(a-\gamma)C & (a-\gamma)M\\
    -(a+r)C & r-d-(a+r)-2MC 
  \end{bmatrix},  \\
  J_2 & = & 
  \begin{bmatrix} 
    \frac{-g}{1-C(t-\tau)} & \frac{-gM(t-\tau)}{(1-C(t-\tau))^2} \\ 
    0 & 0
  \end{bmatrix}.
\end{eqnarray*}

\end{frame}

\begin{comment}

\begin{frame}\frametitle{Putting Jacobians to Use}
{ Suppose $\begin{bmatrix} x\\y\end{bmatrix}=\overrightarrow{v_1}e^{\lambda t}$.}\\\vspace{2em} 
{ Differentiating, setting equal to 1, and doing algebra stuff, we get $(J_1+J_2e^{-\lambda t} -\lambda I)\overrightarrow{v_1}=0$.} 
\end{frame}
\end{comment}

\begin{frame}\frametitle{Putting Jacobians to Use}
Consider the Jacobians evaluated at the origin:

\begin{eqnarray*}
  J_1 & = & 
            \begin{bmatrix}
              \gamma & 0\\
              0 & r-d
            \end{bmatrix}, \\
  J_2 & = & 
            \begin{bmatrix}
              -g & 0\\
              0 & 0
            \end{bmatrix}.
\end{eqnarray*}

\end{frame}


\begin{frame}[c]\frametitle{Putting Jacobians to Use}
%Recall, $(J_1+J_2e^{-\lambda t} -\lambda I)\overrightarrow{v_1}=0$. \\\vspace{2em} Taking determinants, we find the characteristic polynomial of $(J_1+J_2e^{-\lambda t} -\lambda I)$ evaluated at $M=0$, $C=0$ to be 

  The characteristic polynomial of
  $(J_1+J_2e^{-\lambda t} -\lambda I)$, is
$$(\lambda - r + d)(\lambda -\gamma + ge^{-\lambda\tau})=0.$$

\end{frame}

\begin{frame}\frametitle{Get Them Eigenvalues}
Hence, 
\begin{itemize}{\itemsep .5in}
\item $\lambda = r-d>0$ is an eigenvalue with positive real part
\item Other eigenvalues satisfy $\lambda=\gamma-ge^{-\lambda\tau}$.
\item For $\tau\geq0$, our system has two eigenvalues with positive
  real part. This implies our nondelay system is unstable at $(0,0)$.
\end{itemize}
\end{frame}

\begin{comment}
\begin{frame}\frametitle{Get Them Eigenvalues}
Now we consider postive $\tau$.
\begin{itemize}
\item Let $\lambda \tau=(\alpha+i\omega)\tau$.
\item Apply Euler's Formula: $\gamma-g(e^{-\lambda\tau})=\gamma-g(\cos(\lambda\tau)-i\sin(\lambda\tau))$.
\end{itemize}
\end{frame}\end{comment}

\begin{frame}
The time delay model fails to account for stochastic features of the ecosystem:
\begin{itemize}
\item grazing habits of parrotfish, 
\item hurricanes.
\end{itemize}
\end{frame}

\begin{frame}
\frametitle{Stochastic Model}
To account for these stochastic feature we add noise!
$$\begin{cases}
\begin{array}{rl}
\frac{dM}{dt}\hspace{-.8em}&=aMC - \frac{gM(t-\tau)}{1-C(t-\tau)}+\gamma M T+\beta M(1-M)dW,\\
\frac{dC}{dt}\hspace{-.8em}&=rTC - dC - aMC.\\
\end{array}
\end{cases}$$
\end{frame}

\begin{frame}{Transition}
  Need a transition slide here.
\end{frame}
