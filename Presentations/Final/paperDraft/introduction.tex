Coral reef ecosystems support a plethora of diverse organisms in the colorful calcium carbonate accretions of the stony coral reef builder and provide a source of rare biomedical and industrial substrates. Of the many organisms in this ecosystem, the growth of coral reef organisms is observed to be dependent upon a variety of species of algae. 

This complex interaction between coral reefs and algae exist with three morphological types of algae being seen as key: microalgae, macroalgae, and algal turf. Microalgae serve as primary producers of O$_2$, simple carbohydrates, amino acids, and CaCO$_3$ for the coral in a symbiotic relationship where the microalgae are provided a habitat (Dietrich, 1985). 

Macroalgae overgrow upon the coral leading to a state of decreased coral reef covering. Despite consumption of dead coral giving scarids their common name, parrotfishes, the main influence scarids have on coral reefs is realized by the grazing of scarids upon these macroalgae. After macroalgae are grazed, algal turf and coral organisms exist in a dynamic equilibrium until macroalgae grow back after a time delay of 6 to 12 months.